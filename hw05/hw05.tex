%Assignment 5 for EECS 314 Spring 2012.

\documentclass{article}

\usepackage[margin=1in]{geometry}
\usepackage{amsmath}

\title{EECS 314 \\ HW \#5}
\author{John Cleaver}
\date{24 April 2012}

\begin{document}
\maketitle

\section*{1}

{\bf a}.
\begin{align*}
AMAT &= Hit Time + Miss Rate * Miss Penalty \\
AMAT &= 2 + .05 * 20 * 2 = 4 ns
\end{align*}

{\bf b}.
\begin{align*}
CPU Time_a &= IC * (2 + 1.5 * 20 * .05) * 2   = 7 * IC \\
CPU Time_b &= IC * (2 + 1.5 * 20 * .03) * 2.4 = 6.96 * IC
\end{align*}

Because the times are almost identical, the there is no gain from doing this.

\section*{2}

\begin{tabular}{|l|l|}
\hline
Reference & Hit or Miss \\ \hline
2 & Miss \\ \hline
3 & Miss \\ \hline
11 & Miss \\ \hline
16 &  Miss \\ \hline
21 &  Miss \\ \hline
13 &  Miss \\ \hline
64 &  Miss \\ \hline
48 &  Miss \\ \hline
19 &  Miss \\ \hline
11 & Hit \\ \hline
3 &  Miss \\ \hline
22 &  Miss \\ \hline
4 &  Miss \\ \hline
27 &  Miss \\ \hline
6 &  Miss \\ \hline
11 &  Miss \\ \hline
\end{tabular}

\begin{tabular}{|l|l|}
\hline
Index & Cache Contents \\ \hline
0 & 48 \\ \hline
1 & \\ \hline
2 & 2 \\ \hline
3 & 3 \\ \hline
4 & 4 \\ \hline
5 & 21 \\ \hline
6 & 6 \\ \hline
7 & \\ \hline
8 & \\ \hline
9 & \\ \hline
10 & \\ \hline
11 & 11 \\ \hline
12 & \\ \hline
13 & 13 \\ \hline
14 & \\ \hline
15 & \\ \hline
\end{tabular}

\section*{3} 

Processor 1:
\begin{align*}
Miss penalty &= 6 + 1 = 7 cycles \\
Stalls per Inst. &= 4\% * 7 + 50\% * 6\% * 7 =.49
\end{align*}

Processor 2:
\begin{align*}
Miss penalty &= 6 + 4 = 10 cycles \\
Stalls per Inst. &= 2\% × 10 + 50\% × 4\% × 10 = 0.4
\end{align*}

Processor 3:
\begin{align*}
Miss penalty &= 6 + 4 = 10 cycles \\
Stalls per Inst. &= 2\% × 10 + 50\% × 3\% × 10 = 0.35
\end{align*}

Processor 1 spends the most time on Cache misses.

\section*{4}

\section*{5} 

Compulsory, capacity and conflict misses should be reduced if the program is written to require less memory. Increased clock speed will not have an effect on misses regardless of type, but associativity should reduce conflict misses.

\section*{6}

\end{document}