%The second Homework assignment for EECS 314 in Spirng 2012.

\documentclass[12pt]{article}

\usepackage[margin=1in]{geometry}
\usepackage{times}
\usepackage{listings}
\usepackage{mips}
\usepackage{amsmath}

\title{EECS 314 \\ HW 02}
\author{John Cleaver \\ jkc22}
\date{16 Feb 2012}

\begin{document}
\lstset{language=[mips]Assembler}
\maketitle

\section*{1.}

%S1 = S1 / 2
%t0 = S0 * 2^31
%S0 = S0 / 2
%S1 = S1 | t0

%Remove LSB of S1
%Remove all but the LSB of S0 and store in t0
%Remove LSB of S0
%t0 = LSB0000000000000000000
%S1 = 0xxxxxxxxxxxxxxxxxxxxx

\section*{2.}

\begin{lstlisting}

add $t0, $zero, $zero 			#t0 = 0+0

loop:
	beq $a1, $zero, finish 		#If a1 = 0, go to finish
	add $t0, $t0, $a0 		#t0 = t0 + a0
	sub $a1, $a1, $a0 		#a1 = a1 - a0
	j loop				#Go to Begining of loop

finish:
	addi $t0, $t0, 100		#t0 = t0 + 100
	add $v0, $t0, $zero		#v0 = t0 + 0

\end{lstlisting}

This program performs the operation $a \times b$. The code could be rewritten using multiplication explicitly as shown below.

\begin{lstlisting}

mult $v0, $a0, $a1

\end{lstlisting}

\section*{3.}

\begin{tabular}{|l|l|}
	\hline
        Pseudoinstruction  & Implementation                                     \\ \hline
        move \$t1, \$t2      & addi \$t1, \$t2, 0                                   \\ \hline
        clear \$t0          & add \$t0, \$zero, \$zero                              \\ \hline
        beq \$t1, small, L  & ~                                                  \\ \hline
        beq \$t2, big, L    & ~                                                  \\ \hline
        li \$t1, small      & lui \$t1, small; srl \$t1, \$t1, 16;                  \\ \hline
        li \$t2, big        & srl \$at, big, 16; lui \$t2, \$at; ori \$t2, \$t2, big; \\ \hline
        ble \$t3, \$t5, L    & slt \$at, \$t5, \$t3; beq \$at, \$zero, L;              \\ \hline
        bgt \$t4, \$t5, L    & slt \$at, \$t5, \$t4; bne \$at, \$zero, L;              \\ \hline
        bge \$t5, \$t3, L    & slt \$at, \$t5, \$t3; beq \$at, \$zero, L;              \\ \hline
        addi \$t0, \$t2, big & ~                                                  \\ \hline
        lw \$t5, big (\$t2)  & ~                                                  \\
	\hline	
\end{tabular}

\section*{4.}

OC-relative branching jumps to a place in memory relative to where the program execution is at the time of branching. Because it is relative, there is a limit to how far in memory the program can jump (in MIPS, it is 16 bits). If `there' is more than 16 bits of memory away from `here', then the branch is not possible. The assembler might correct this by having an intermediate jump statement that is less than 16 bits away from `here' and that jumps to `there'.

\section*{5.}

\section*{6.}

\section*{7.}

\noindent{\bf a.} $-5435391$

\noindent{\bf b.} $2903506946$

\noindent{\bf c.} $2.903506946^9$

\noindent{\bf d.} It is an sw operation in MIPS.

\section*{9.}

\section*{10.}

Given that the CPI of the machine is the sum of the weighted CPIs for each set of instructions:

\begin{align*}
	CPI_{mbase} &= (2 * 40 + 3 * 25 + 3 * 25 + 5 * 10) / 100 = 2.80 \\
	CPI_{mopt} &= (2 * 40 + 2 * 25 + 3 * 25 + 4 * 10) / 100 = 2.45
\end{align*}

This would give the $CPI_{mopt}$ a slight advantage in performance.

\end{document}
